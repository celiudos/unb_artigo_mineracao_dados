\section{Introdução}

A proteção do trabalho da mulher é assegurada pela Consolidação das Leis Trabalhistas (CLT), de 1943. Além disso, existe a Lei 9.029, de 1999, que instituiu regras sobre o acesso da mulher ao mercado de trabalho. Esse dispositivo proíbe, por exemplo, o anúncio de vagas de emprego com referência ao sexo ou que o sexo da pessoa seja uma variável determinante para fins de remuneração e oportunidades de ascensão profissional. 

No entanto, a diferença de remuneração entre homens e mulheres, que vinha em tendência de queda até 2020, voltou a subir no país e atingiu 22\% no fim de 2022, segundo dados do Instituto Brasileiro de Geografia e Estatística (IBGE). Isso significa que uma brasileira recebe, em média, 78\% do que ganha um homem \cite{dif_salarial_CNN}.

Segundo pesquisas realizadas pelo McKinsey Global Institute  \cite{covid_affeted_gender}, a pandemia afetou desproporcionalmente as mulheres no mercado de trabalho. Muitas mulheres deixaram seus empregos para cuidar de suas famílias e lares durante esse período difícil, o que pode ter contribuído para o aumento recente na diferença salarial entre homens e mulheres. O economista Bruno Imaizumi sugere que "as mulheres podem ter ficado mais tempo fora do mercado de trabalho, tornando mais difícil sua reinserção".

Dentro deste contexto, propõe-se verificar se essa desigualdade salarial é também percebida entre os profissionais da área de Tecnologia e Informação (TI) em todo o território brasileiro. Se sim, identificar qual a sua proporção.

O trabalho está organizado da seguinte forma: Seção II são descritos conceitos do RAIS, sobre a pandemia e seus efeitos nas mulheres; Seção III apresenta os trabalhos correlatos; na Seção IV a metodologia utilizada e os passos para exibir os dados estatísticos; na Seção V os resultados obtidos; na Seção VI as conclusões e trabalhos futuros.