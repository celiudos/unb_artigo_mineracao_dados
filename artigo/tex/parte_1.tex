\section{Introdução}

Com o avanço exponencial das tecnologias modernas, que estão modificando rapidamente as indústrias e sociedades globalmente, espera-se que a maneira como as pessoas trabalham, vivem e interagem com as outras seja transformada em uma velocidade e escala sem precedentes históricos \cite{hand1981artificial}. As inovações tecnológicas estão realmente mudando velozmente a fronteira entre as tarefas executadas por pessoas e por máquinas , transformando o mundo do trabalho \cite{aksoy2021robots}. O que não se sabe é se essas transformações estão também remodelando aspectos socioeconômicos, como as diferenças salariais entre gêneros.

A proteção do trabalho da mulher é assegurada pela Consolidação das Leis Trabalhistas (CLT), de 1943. Além disso, existe a Lei 9.029, de 1999, que instituiu regras sobre o acesso da mulher ao mercado de trabalho. Esse dispositivo proíbe, por exemplo, o anúncio de vagas de emprego com referência ao sexo ou que o sexo da pessoa seja uma variável determinante para fins de remuneração e oportunidades de ascensão profissional. No entanto, a diferença de remuneração entre homens e mulheres, que vinha em tendência de queda até 2020, voltou a subir no país e atingiu 22\% no fim de 2022, segundo dados do Instituto Brasileiro de Geografia e Estatística (IBGE). Isso significa que uma brasileira recebe, em média, 78\% do que ganha um homem.

Do ponto de vista de \cite{ahmed2015human}, é evidente a discriminação no mercado de trabalho contra as mulheres que, quando possuem as mesmas características dos homens em relação ao capital humano e desempenham a mesma atividade, recebem remuneração diferente em razão de seu sexo. Além disso, o relatório de 2015 da Organização das Nações Unidas (ONU) mostra que a assimetria salarial de sexo em favor dos homens ocorre em todas as indústrias e ocupações na maioria dos 177 países observados \cite{report2015onu}.

Dentro desse contexto, propõe-se verificar, neste estudo, se essa desigualdade salarial é também percebida entre os profissionais da área de Tecnologia da Informação (TI) em todo o território brasileiro, além de identificar se há uma diferença na quantidade de demissões. Para isso, utilizou-se dados da Relação Anual de Informações Sociais (RAIS) de 2015 a 2021, que fornece dados oficiais sobre o mercado de trabalho no Brasil. 

O trabalho está organizado da seguinte forma: Seção II, que trata do referencial teórico; Seção III, que apresenta os trabalhos correlatos; na Seção IV, a metodologia utilizada e os passos para exibir os dados estatísticos; na Seção V são apresentados os resultados obtidos e, pro fim, na Seção VI está a conclusão.