\section{Introdução}

::: Artigo - PROFISSIONAIS DE TI NO NORDESTE EM UM CONTEXTO DE CRISE PROLONGADA :::

Este artigo objetiva analisar o novo segmento da tecnologia da informação (TI) no Nordeste brasileiro a partir das transformações e tendências trazidas pelos impactos da crise econômica desencadeada em 2015 e seu agravamento com a pandemia da Covid-19. Especialmente, se detém sobre indicadores da condição laboral dos profissionais de TI na região. 

A tecnologia da informação, peça-chave do processo de reestruturação produtiva que se propagou com a crise do fordismo e a globalização (Castells, 1996), tornou-se cada vez mais central para todos os ramos da indústria e dos serviços: alterou as dinâmicas produtivas, demandou adequações nas estruturas legais e regulatórias da economia e modificou os termos da divisão técnica do trabalho e as qualificações, competências e habilidades exigidas dos trabalhadores e suas formas de contratação, trazendo fortes implicações para o mundo do trabalho. A partir disso, derivou o que passou a ser designado por Economia da Informação (Castells, 1996), envolvendo ramos econômicos cada vez mais diversos, entre os quais os segmentos de produção de hardware de software, de telecomunicações, de coleta, armazenamento e análise da informação etc. 

Baseando-se em tipologia formulada por Bukht e Heeks (2017), o relatório da United Nations Conference on Trade and Development  UNCTAD (2017) adota uma representação da economia digital dividida em três segmentos: o primeiro, denominado de broad scope (ou digitalized economy), compreende e-business, e-commerce, indústria 4.0, agricultura de precisão e economia algorítmica; o segundo, designado por narrow scope (ou digital economy), envolve serviços digitais, economia de plataforma, economia de compartilhamento e gig economy; e o terceiro, a core (digital sector  TI/TIC), que inclui produção de hardware, software e consultoria em TI, serviços de informação e telecomunicações \cite{Sobre_o_MPF}.

O trabalho está organizado da seguinte forma: Seção II são descritos conceitos do Portal da Transparência do MPF e Consulta Processual, Apache Parquet, Neo4J e Cypher Query; Seção III apresenta os trabalhos correlatos; na Seção IV a Metodologia Utilizada, Arquitetura de Hardware e Software, Coleta e limpeza dos dados e os Dados gerados; na Seção V a Comparação de leitura e escrita de arquivo no Pandas, Análise otimizada de grande volume de dados utilizando amostra estratificada, e Análise de Partes em Processos Judiciais de 2022 (ano eleitoral); na Seção VI as conclusões.