\section{Introdução}

A pandemia da COVID-19 teve um impacto significativo na forma como as instituições, tanto do setor público como do setor privado, operam e interagem com seus clientes e cidadãos, respectivamente. A pandemia causou uma aceleração da digitalização dos serviços dessas instituições, o que alargou a demanda por profissionais de TI qualificados para implementar e gerenciar soluções tecnológicas. Em outras palavras, a pandemia destacou a importância da tecnologia e dos profissionais de TI na manutenção da continuidade dos negócios e na adaptação a um ambiente em rápida mudança. Entretanto, segundo muitos chefes dos setores de TI do serviço público, o investimento em soluções tecnológicas aumentou, mas a valorização dos profissionais de TI não cresceu na mesma proporção. 
		
Essa problemática pode ser mais bem percebida no site https://www.gaptic.com.br/ que trata sobre a Gratificação de Atividade de Tecnologia da Informação no âmbito do Ministério Público Brasileiro e do Judiciário Brasileiro. O secretário da STIC do TRE/PE, George Maciel, disse, no Encontro Nacional de Tecnologia e Inovação da Justiça Federal de abril de 2023: “Estamos perdendo pessoas, as informações estão indo embora. Não conseguimos reter esse conhecimento. E aí vem a pergunta: como queremos fazer toda essa transformação digital sem investir nos servidores de TI?”
		
Nesse contexto de falta de valorização dos profissionais de TI no setor público e consequente rotatividade, o grupo de trabalho propõe uma pesquisa para verificar a existência de impacto da COVID-19 sobre o salário dos profissionais de TI e se ela provocou uma discrepância entre o setor público e o setor privado, com o intuito de subsidiar a criação de políticas de valorização e retenção de empregados e servidores públicos.