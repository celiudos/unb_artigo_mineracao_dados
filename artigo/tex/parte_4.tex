\section{Trabalhos Relacionados}

Em 2022 foi realizado um estudo \cite{oliveira2022profissionais} que consistia em analisar novo segmento da Tecnologia da Informação (TI) no Nordeste brasileiro a partir das transformações e tendências trazidas pelos impactos da crise econômica desencadeada em 2015 e seu agravamento com a Pandemia da Covid-19. Este trabalho se deteve, especialmente, a analisar indicadores da condição laboral dos profissionais de TI na região. Para isso, utilizou-se estudos empíricos e dados secundários, provenientes da Pesquisa Anual de Serviços (PAS) e da Pesquisa Nacional por Amostra de Domicílios Contínua (PNADC), ambas do Instituto Brasileiro de Geografia e Estatística (IBGE), e, com maior destaque, a Relação Anual de Informações Sociais (RAIS). Os dados evidenciaram que, tanto no plano nacional como regional, a crise exerceu um efeito negativo sobre esse segmento, apesar do seu estoque de vínculos formais ter continuado sua trajetória de crescimento. Nesse trbalho não foi analisada a diferença salarial entre homens e mulheres na área de TI.

Em 2021, foi realizado um estudo por Rodolfo Bechtlufft e Bruno Lazarotti \cite{bechtlufft2021determinantes} que analisou os determinantes do diferencial de remuneração entre as carreiras do Poder Executivo do estado de Minas Gerais. Os autores utilizaram um modelo de regressão linear cujas variáveis explicativas foram elaboradas com base em diferentes abordagens teóricas acerca da determinação dos salários: teoria do capital humano, teoria da segmentação dos mercados, teoria da discriminação e sociologia das profissões, além de considerar as especificidades do setor público 1. Os resultados sugerem que a estrutura relativa de salários no setor público constitui uma manifestação concreta das diferenças de recursos políticos disponíveis às carreiras, do prestígio social das profissões e de aspectos estruturais da desigualdade de gênero, mas, diferentemente do que é proposto neste trabalho, o estudo focou apenas nas carreiras do poder executivo do estado de MG.