\section{Metodologia}

Todo o estudo proposto será realizado seguindo a metodologia Cross Industry Standard Process for Data Mining - CRISP-DM \cite{chapman2000crisp}, que é um processo padrão para mineração de dados.  Cabe destacar que não é proprietário e pretende ser independente do setor e das aplicações em que é utilizado.

O CRISP-DM envolve um ciclo faseado para um projeto ou pesquisa de mineração de dados. As atividades que compreendem este artigo serão realizadas conforme as seguintes etapas do CRISP-DM a seguir:

\begin{enumerate}
	\item Compreensão do negócio: Análise detalhada do tema "diferença de gênero" e a área de TI.
	\item Compreensão dos dados: os dados do RAIS serão coletados e examinados para identificar problemas de qualidade e verificar se eles são adequados para atender aos objetivos da pesquisa. Além disso, serão identificadas as variáveis que serão utilizadas na análise.
	\item Preparação dos dados: os dados da RAIS serão preparados para análise, o que pode incluir a limpeza de dados, a transformação de variáveis e a seleção de subconjuntos de dados relevantes. 
	\item Modelagem: serão aplicados modelos estatísticos para identificar padrões nos dados da RAIS, bem como relações entre variáveis.
	\item Avaliação: os resultados da modelagem serão avaliados para verificar se eles atendem aos objetivos da pesquisa, isto é, se eles fornecem informações úteis.
	\item Implantação: nesta etapa final, os resultados serão apresentados e as hipóteses confirmadas ou refutadas. 
\end{enumerate}
	      	      	      
\subsection{Compreensão dos dados}
	      	      	    
A análise foi feita com foco em quatro grupos, conforme a lista abaixo:

\begin{itemize}
	\item Nível educacional 	      	      	      	      	      
	\item Setor privado e público	      	      	      	      	     
	\item Quantidade de demissões    	      	      	      	     
	\item Diferença entre cargos de TI, conforme a tabela CBO \ref{ocupacoes}    	      	      	     
\end{itemize}
	      	      	    
Cada grupo foi analisado com base em duas variáveis:

\begin{itemize}
	\item Quantidade de pessoas do sexo masculino e feminino	      	      	      	      	      
	\item Salário médio ao longo de um ano	    
\end{itemize}
	      	      	      
A seguir, resumimos os dados analisados em uma tabela \ref{vars}:     

\begin{table}[htbp]
	\caption{Dados analisados}
	\begin{center}
		\begin{tabular}{|c|c|c|}
			\hline
			\textbf{Variável}           & \textbf{Definição}      & \textbf{Exemplo}    \\ 
			\hline 
			\textbf{Ano}                 & Período dos dados        & 2018, 2019          \\
			\hline
			\textbf{Sexo}                & Masculino ou Feminino     & 1 ou 2              \\
			\hline
			\textbf{Salário}            & Com base no ano           & 5000                \\
			\hline 
			\textbf{Nível educacional}  & Nível médio e superior  & Mestrado            \\
			\hline 
			\textbf{Tipo de vínculo}    & Setor Privado ou Público & CLT, Estatutário   \\
			\hline 
			\textbf{Motivo Desligamento} & Demissão, etc            & Com acordo, sem     \\
			\hline
			\textbf{Ocupação}          & Descrição do cargo      & Analista de suporte \\
			\hline 
		\end{tabular}
		\label{vars}
	\end{center} 
\end{table}      	      

\subsection{Preparação dos dados}

Foi feito um processo de Extração, Transformação e Carga (ETL) para preparar os dados para análise. O processo foi feito com a linguagem Python e a análise com a biblioteca Pandas. A fonte de dados da RAIS foi extraída a partir do site basedosdados.org \cite{basedosdados}, utilizando a API disponível através da plataforma BigQuery \cite{bigquery}. Os dados extraídos compreendem o período de 2015 a 2021, pois foram os dados disponibilizados para este estudo sem custos adicionais da plataforma BigQuery. 

\subsection{Modelagem}

A modelagem foi feita com base em gráficos de barras, de linhas, e de mapas. Os gráficos de barras e de linhas foram utilizados para mostrar a quantidade de pessoas do sexo masculino e feminino, bem como o salário médio ao longo de um ano. Os mapas foram utilizados para mostrar a quantidade de pessoas do sexo masculino e feminino por estado.