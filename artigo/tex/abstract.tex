\begin{abstract}
	Este estudo aborda a diferença salarial entre homens e mulheres na área de TI durante a pandemia. Ele também explora possíveis cenários para analisar a disparidade salarial e de desligamento entre gêneros na área de tecnologia, antes e após a pandemia. Utilizou-se dados no nível do indivíduo, de 2018 a 2021, obtidos da Relação Anual de Informações Sociais (Rais), que proporciona dados oficiais sobre o mercado de trabalho no Brasil. No geral, constatou-se que a remuneração média das mulheres é menor que a dos homens em todas as regiões do Brasil, tanto em 2019 quanto em 2020. Além disso, a quantidade de desligamento de homens é maior que a de mulheres em todas as regiões do Brasil, tanto em 2019 quanto em 2020.
\end{abstract}

\begin{IEEEkeywords}
	Tecnologia da informação. Profissionais de TI. Diferença salarial. Gênero. Pandemia. COVID-19.
\end{IEEEkeywords}