\begin{abstract}
	Este artigo objetiva analisar o novo segmento da Tecnologia da Informação (TI) no Brasil  a partir das transformações e tendências trazidas com a Pandemia da Covid-19. Este trabalho se detém, especialmente, sobre indicadores da condição laboral dos profissionais de TI na região. Para isso, utiliza-se de estudos empíricos e de dados secundários. Quanto a estes últimos, priorizamos a Relação Anual de Informações Sociais (Rais). Os dados evidenciaram que, tanto no plano nacional como regional, a crise exerceu um efeito negativo sobre esse segmento, apesar do seu estoque de vínculos formais ter continuado sua trajetória de crescimento.
\end{abstract}

\begin{IEEEkeywords}
	Tecnologia da informação e comunicação. Profissionais de TI. Crise econômica. Brasil.
\end{IEEEkeywords}