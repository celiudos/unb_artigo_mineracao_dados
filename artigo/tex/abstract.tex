\begin{abstract}
	Este estudo aborda a diferença salarial entre homens e mulheres na área de Tecnologia da Informação (TI) durante a pandemia. Ele também explora possíveis cenários para analisar a disparidade salarial e de desligamento entre gêneros na área de tecnologia, antes e após a pandemia. Utilizou-se dados no nível do indivíduo, de 2015 a 2021, obtidos da Relação Anual de Informações Sociais (Rais), que proporciona dados oficiais sobre o mercado de trabalho no Brasil. No geral, constatou-se que, além da quantidade de profissionais mulheres ser bem menor que a de homens, o mesmo comportamento acontece para a média salarial. Sob esses dois aspectos, vemos os dados diante do nível educacional, setor privado e público, demissões e entre cargos de TI. Em todos os cenários, as diferenças são bem maiores em 2021, ano em que a pandemia estava em vigor.
	
\end{abstract}

\begin{IEEEkeywords}
	Tecnologia da informação. Profissionais de TI. Pandemia. Gênero. Diferença salarial. Diferença quantitativa.
\end{IEEEkeywords}