\begin{abstract}
	O Ministério Público Federal (MPF) possui um sistema principal com mais 88 milhões de processos disponíveis. Conforme esse volume aumenta, fica mais difícil de entender a relação entre os processos. Este estudo engloba os desafios de se utilizar dados orientados a grafo, utilizando o banco de dados Neo4J. Nesta análise foi verificada a relação entre documentos e partes contidos em um Procedimento Extrajudicial não sigiloso. Durante a realização deste estudo foi feita uma análise de 2014 até 2023 e outra de apenas 2022, ano eleitoral. No primeiro recuperamos aproximadamente 1.2 milhões de documentos, enquanto no segundo recuperamos por volta de 117 mil Partes e seus respectivos Procedimentos. Foi possível verificar que a utilização de arquivo otimizados para ciência de dados, como o Parquet no Pandas apresenta um desempenho melhor no tempo de processamento. Por fim, constatamos que a utilização de amostra estratificada é mais eficiente e pode ser uma estratégia para agilizar as etapas iniciais do processo de análise de dados. Por fim, visualizamos em grafo  alguns exemplos de dados do ano eleitoral de 2022.
\end{abstract}

\begin{IEEEkeywords}
	Ministério Público; Banco de Dados Massivos; Neo4J; Cypher; Grafo; Pandas; Python; Visualização de Dados; Amostra Estratificada.
\end{IEEEkeywords}