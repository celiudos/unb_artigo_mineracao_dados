\section{Methodology}

All the proposed study will be carried out following the Cross Industry Standard Process for Data Mining - CRISP-DM methodology \cite{chapman2000crisp}, which is a standard process for data mining. It should be noted that it is not proprietary and intends to be independent of the sector and applications in which it is used.

CRISP-DM involves a phased cycle for a data mining project or research. The activities that comprise this article will be carried out according to the following CRISP-DM stages:

\begin{enumerate}
	\item Business understanding: the context of the presented problem was analyzed in detail for a better understanding of the "gender difference" theme and the importance of the Information Technology area.
	\item Data understanding: RAIS data was collected and examined to identify quality problems and verify if they are adequate to meet research objectives. In addition, the variables that would be used in the analysis were identified.
	\item Data preparation: RAIS data was prepared for analysis. This included data cleaning, variable transformation, and selection of relevant data subsets
	\item Modeling: statistical techniques were applied to identify patterns in RAIS data, as well as relationships between variables.
	\item Evaluation: the modeling results were evaluated to verify if they met research objectives, that is, if they provided useful information.
	\item Deployment: in this final stage, the results were presented and hypotheses confirmed or refuted.
\end{enumerate}
	      	      	      
\subsection{Business understanding}
	      	      	    
The analysis was focused on four groups, as listed below:

\begin{itemize}
	\item Educational Level	      	      	      	      
	\item Private and Public Sector	      	     
	\item Layoffs
	\item Difference between IT jobs, as per CBO t{ocupacoes}    	      	      	     
\end{itemize}
	      	      	    
Each group was analyzed based on two variables:

\begin{itemize}
	\item Number of male and female persons	      	      	      	      	      
	\item Average salary over each year from 2015 to 2021.	    
\end{itemize}
	      	      	      
In the following Table \ref{vars} is a summary of the analyzed data:     

\begin{table}[htbp]
	\caption{Analyzed data}
	\begin{center}
		\begin{tabular}{p{2cm}|p{3cm}|p{2cm}}
			\hline
				\textbf{Variable} & \textbf{Definition} & \textbf{Example}    \\
			\hline 
				\textbf{Year} & Data Period & 2018, 2019 \\
			\hline
				\textbf{Gender} & Male or Female & 1 ou 2 \\
			\hline
				\textbf{Salary} & On a yearly basis & 5000 \\
			\hline 
				\textbf{Education Level}  &  High School and Higher education &Masters Degree \\
			\hline 
			\textbf{Employment relation}    & Private or Public Sector  & CLT, Estatutário   \\
			\hline 
			\textbf{Reason for layoffs} & Resignation, etc & w/ severance agreement, w/o     \\
			\hline
			\textbf{Occupation}          & Job Description      & Support Analyst \\
			\hline 
		\end{tabular}
		\label{vars}
	\end{center} 
\end{table}      	      

\subsection{Data preparation}

An Extraction, Transformation and Loading (ETL) process was performed to prepare the data for analysis. The process was done using the Python language and the analysis with the Pandas library. The RAIS data source was extracted from the basedosdados.org website \cite{basedosdados}, using the API available through the BigQuery platform \cite{bigquery}. The extracted data covers the period from 2015 to 2021, as these were the data made available for this study at no additional cost from the BigQuery platform.

\subsection{Modeling}

The modeling was done based on bar, line and map charts. Bar and line charts were used to show the number of male and female people, as well as the difference in average salary by gender for each year. Maps were used to show both the distribution of the number of male and female professionals, as well as the distribution of the difference in average salary by gender throughout Brazilian territory. The difference in average salary was calculated by dividing the average salary of females by the average salary of males and subtracting the result by 1. This difference is given in percentage terms. The same formula was used to calculate the difference in quantity by gender for each year and the difference in the number of dismissals by gender for each year.
