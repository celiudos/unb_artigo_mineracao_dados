\section{Related Works}

In the article \cite{de2021evidencias}, the authors analyze the gender gap in the Brazilian manufacturing industry for different quantiles of men's and women's wage distribution. In all productivity strata, an unfavorable income differential was found for women.
The results also suggest that women with high qualifications would be attracted to more productive sectors; however, as observed, these same sectors have low female representation in management positions, which commonly offer higher incomes.

One of the most recent studies conducted on the subject was Ribeiro's (2022) \cite{ribeiro2022efeitos}. In her doctoral thesis, the author found that there is a salary increase for both genders, being 4.6\% for men and 5.2\% for women, when employed in companies with high intensity of use of IT compared to companies with low intensity of use. The study focused on the data of the company's economic activity and its technologies used. This work pointed out that the effects of IT on wage inequality are heterogeneous, with distinct impacts by educational level but not by gender.

In Brazil, a study conducted in the southern region of the country \cite{camargo2019relaccoes} analyzed power relations between genders and female representation in information technology (IT) companies in Porto Alegre - RS. It was found that, despite certain advances in gender equity, inequality was still present in organizations. The results showed that there are still significant gender differences in the IT field, both in academia, which leads to the majority of women dropping out, and in organizations.

Hoisl and Mariani's study (2017) \cite{hoisl2017sa} analyzed wage disparities as the basis for determinants of gender imbalances in the labor market. The study found that, even with training and education as relevant factors, women earn less than men, even though they contribute to the development of high-quality inventions as much as men.

This study differs from others by analyzing the salary and layoffs difference between men and women who work specifically in the IT field, regardless of the organization's industry or whether it is public or private sector, in the period between 2015 and 2021. The study presents a gender diagnosis in the Brazilian territory, looking at variables such as salary, layoffs, and education.
