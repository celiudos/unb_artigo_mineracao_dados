\section{Resultados Obtidos}

As análises foram feitas entre os anos de 2019 (pré COVID) e 2020 (pós COVID). Conforme o cenário mundial mudou de um ano para o outro, percebemos que as diferenças entre os gêneros também mudaram.

\subsection{Resultados de 2019}

Conforme a tabela abaixo \ref{remun}, a remuneração média das mulheres é menor que a dos homens em todas as regiões do Brasil. A diferença é maior na região Sudeste, onde a remuneração média das mulheres é 38\% menor que a dos homens. A região Norte apresenta a menor diferença, com 4\% menor.

\begin{table}[htbp]
	\caption{Remuneração}
	\begin{center}
		\begin{tabular}{|c|c|c|c|c|c|}
			\hline
			\textbf{Região} & \textbf{Sexo} & \textbf{Média} & \textbf{Min.} & \textbf{Máx.} & \textbf{DP} \\ 
			\hline
																																																																			
			Centro-Oeste     & Feminino      & 2726.01         & 387.47        & 17874.86       & 2544.76     \\
			                 & Masculino     & 3653.59         & 316.66        & 41448.09       & 3056.77     \\
			\hline
			Nordeste         & Feminino      & 2848.94         & 345.04        & 29940.00       & 2560.51     \\
			                 & Masculino     & 3677.29         & 331.76        & 56616.23       & 3702.60     \\
			\hline
			Norte            & Feminino      & 3390.81         & 309.67        & 110716.20      & 4902.86     \\
			                 & Masculino     & 3260.72         & 301.13        & 105844.80      & 4172.60     \\
			\hline
			Sudeste          & Feminino      & 5271.16         & 300.63        & 50491.29       & 4383.12     \\
			                 & Masculino     & 5377.40         & 300.00        & 120828.57      & 4736.53     \\
			\hline
			Sul              & Feminino      & 3905.04         & 303.33        & 45743.98       & 3616.55     \\
			                 & Masculino     & 4250.18         & 304.16        & 84126.00       & 3399.41     \\
			\hline
		\end{tabular}
		\label{remun}
	\end{center}
\end{table}

Já quanto às demissões em 2019, a tabela abaixo \ref{demi} nos mostra que a quantidade de homens desligados é maior que a de mulheres em todas as regiões do Brasil. A região Sudeste apresenta a maior diferença, com 80\% maior.

\begin{table}[htbp]
	\caption{Demissões}
	\begin{center}
		\begin{tabular}{|c|c|c|}
			\hline
			\textbf{Região} & \textbf{Sexo} & \textbf{Qnt. deslig.} \\ 
			\hline																
			Centro-Oeste     & Feminino      & 204                   \\
			                 & Masculino     & 2679                  \\
			\hline
			Nordeste         & Feminino      & 1796                  \\
			                 & Masculino     & 9765                  \\
			\hline
			Norte            & Feminino      & 801                   \\
			                 & Masculino     & 1722                  \\
			\hline
			Sudeste          & Feminino      & 14399                 \\
			                 & Masculino     & 70987                 \\
			\hline
			Sul              & Feminino      & 3470                  \\
			                 & Masculino     & 26534                 \\
			\hline
		\end{tabular}
		\label{demi}
	\end{center}
\end{table}
  

\subsection{Resultados de 2020}

Agora para o ano de 2020, na período da COVID-19, vemos na tabela  \ref{remun20} que a remuneração média das mulheres é menor que a dos homens em todas as regiões do Brasil. A diferença é maior na região Sudeste, onde a remuneração média das mulheres é 40\% menor que a dos homens. A região Norte apresenta a menor diferença, com 8\% menor.

\begin{table}[htbp]
	\caption{Remuneração}
	\begin{center}
		\begin{tabular}{|c|c|c|c|c|c|}
			\hline
			\textbf{Região} & \textbf{Sexo} & \textbf{Média} & \textbf{Min.} & \textbf{Máx.} & \textbf{DP} \\ 
			\hline																																							
			Centro-Oeste     & Feminino      & 2709.50         & 315.69        & 36437.56       & 3365.90     \\
			                 & Masculino     & 3070.05         & 311.70        & 45579.51       & 2661.07     \\
			\hline	
			Nordeste         & Feminino      & 3161.25         & 313.86        & 43623.47       & 3513.94     \\
			                 & Masculino     & 3481.98         & 311.70        & 54084.93       & 3393.45     \\
			\hline	
			Norte            & Feminino      & 3741.41         & 337.10        & 51863.85       & 3419.64     \\
			                 & Masculino     & 3161.25         & 313.59        & 77710.46       & 3429.92     \\
			\hline	
			Sudeste          & Feminino      & 4386.02         & 312.84        & 70360.39       & 4287.26     \\
			                 & Masculino     & 5285.72         & 311.94        & 120581.45      & 5035.27     \\
			\hline	
			Sul              & Feminino      & 3275.24         & 313.82        & 58268.44       & 2953.85     \\
			                 & Masculino     & 3942.60         & 312.05        & 62973.81       & 3347.73     \\
			\hline
		\end{tabular}
		\label{remun20}
	\end{center}
\end{table}

Já quanto às demissões em 2020, a tabela abaixo \ref{demi20} nos mostra que a quantidade de desligamento (Qnt. deslig.) de homens é maior que a de mulheres em todas as regiões do Brasil. A região Sudeste apresenta a maior diferença, com 80\% maior.

\begin{table}[htbp]
	\caption{Demissões}
	\begin{center}
		\begin{tabular}{|c|c|c|}
			\hline
			\textbf{Região} & \textbf{Sexo} & \textbf{Qnt. deslig.} \\ 
			\hline																
			Centro-Oeste     & Feminino      & 1014                  \\
			                 & Masculino     & 4892                  \\
			\hline
			Nordeste         & Feminino      & 3939                  \\
			                 & Masculino     & 17567                 \\
			\hline
			Norte            & Feminino      & 4031                  \\
			                 & Masculino     & 4085                  \\
			\hline
			Sudeste          & Feminino      & 24390                 \\
			                 & Masculino     & 10350                 \\
			\hline
			Sul              & Feminino      & 7720                  \\
			                 & Masculino     & 30439                 \\
			\hline
		\end{tabular}
		\label{demi20}
	\end{center}
\end{table}