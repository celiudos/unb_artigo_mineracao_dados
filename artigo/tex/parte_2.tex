\section{Referêncial Teórico}
\subsection{Relação Anual de Informações Sociais (RAIS)}

A gestão pública brasileira, no setor trabalhista, possui um importante instrumento de coleta de dados denominado RAIS\footnote{http://www.rais.gov.br/sitio/sobre.jsf}. Ele proporciona dados oficiais sobre o mercado de trabalho brasileiro e é mantido pelo Ministério da Economia. O objetivo da RAIS é suprir as necessidades de controle da atividade trabalhista no País, prover dados para a elaboração de estatísticas do trabalho e disponibilizar informações do mercado de trabalho às entidades governamentais. A grande vantagem de utilizar esse instrumento é que,além dele fornecer dados oficiais, as empresas são legalmente obrigadas a enviar seus dados anualmente, incluindo os de seus empregados.

\subsection{Classificação Nacional de Atividades Econômicas (CNAE) e Classificação Brasileira de Ocupações (CBO)}

A CNAE\footnote{https://concla.ibge.gov.br/} é a classificação oficialmente adotada pelo Sistema Estatístico Nacional e pelos órgãos federais gestores de registros administrativos. Tem como aplicações: sistema estatístico, cadastro central de empresas, pesquisas econômicas estruturais, entre outros. Já a CBO\footnote{https://empregabrasil.mte.gov.br/76/cbo/} é um documento que retrata a realidade das profissões do mercado de trabalho brasileiro. 

Os códigos dos grupos CNAE relacionados às atividades de TI são: Atividades dos serviços de tecnologia da informação (620) e Tratamento de dados, hospedagem na Internet e outras atividades relacionadas (631). A partir desses grupos, foi possível encontrar as respectivas ocupações, conforme mostrado na Tabela \ref{ocupacoes} a seguir.

\begin{table}[htbp]
	\caption{Ocupações analisadas}
	\begin{center}
		\begin{tabular}{|c|c|}
			\hline
			\textbf{Código} & \textbf{Descrição}                                  \\ 
			\hline
			212205           & Engenheiro de Aplicativos em Computacao               \\
			212210           & Engenheiro de Equipamentos em Computacao              \\
			212215           & Engenheiros de Sistemas Operacionais em Computacao    \\
			\hline 										
			212305           & Administrador de Banco de Dados                       \\
			212310           & Administrador de Redes                                \\
			212315           & Administrador de Sistemas Operacionais                \\
			212320           & Administrador em Segurança da Informação           \\
			\hline 									
			212405           & Analista de Desenvolvimento de Sistemas               \\
			212410           & Analista de Redes e de Comunicacao de Dados           \\
			212415           & Analista de Sistemas de Automacao                     \\
			212420           & Analista de Suporte Computacional                     \\
			\hline 									
			317105           & Programador de Internet                               \\
			317110           & Programador de Sistemas de Informacao                 \\
			317115           & Programador de Maquinas - Ferramenta com Comando Num. \\
			317120           & Programador de Multimidia                             \\
			\hline 									
			317205           & Operador de Computador (Inclusive Microcomputador)    \\
			317210           & Tecnico de Apoio ao Usuario de Informatica (Helpdesk) \\
			\hline
		\end{tabular}
		\label{ocupacoes}
	\end{center}
\end{table}

\subsection{Salário}

O salário é a remuneração que o empregado recebe pelo trabalho que realiza em uma instituição, pública ou privada. Segundo \cite{kryscynski2021firm}, os trabalhadores e suas empresas se envolvem em recorrentes relações de troca; de um lado, o trabalhador entrega valor, de outro, uma das recompensas que a empresa oferece em troca é o salário.
Considerando a relevância dessa temática na sociedade, é importante analisar a 
 desigualdade salarial, sobreduto entre grupos com alta dispersão salarial, seja de nível educacional, raça, gênero etc. Este trabalho apresentará um diagnóstico com ênfase no gênero dentro dos vários cargos da área de TI, conforme apresentado na tabela \ref{ocupacoes}.

 \subsection{Tecnologia da Informação}

A Tecnologia da Informação (TI) é um conjunto de atividades que envolve o uso de tecnologias para obtenção, armazenamento, acesso, processamento, análise e disseminação de informações. A TI é uma área que está em constante crescimento e evolução; ela tem se tornado cada vez mais essencial para as organizações, sendo um dos principais fatores de competitividade, uma vez que pode ser utilizada para melhorar a eficiência e a eficácia dos processos  organizacionais, além de possibilitar a criação de novos produtos e serviços. Com essa importância, é fácil perceber que as tecnologias avançam exponencialmente a uma velocidade e escala sem precedentes na história. Por ser sinônimo de modernidade, a TI é uma área que atrai muitas pessoas, principalmente os jovens, que estão em busca de uma carreira promissora. No entanto, a área de tecnologia é fundamentalmente marcada por trabalhadores do sexo masculino tanto no mundo, como no Brasil \cite{de2021evidencias}, \cite{nunes2016genero}.

Ainda segundo \cite{de2021evidencias}, uma das explicações para o hiato salarial em relação a gênero é a baixa participação feminina em ocupações ligadas às ciências, engenharias e computação. As pessoas que ocupam cargos no setor STEM (do inglês: Science, Techonology, Engineering and Math) tendem a ter melhores salários do que a média. A presença de engenheiros e cientistas eleva a produtividade das organizações e aumenta a remuneração destes. Se as mulheres têm participação relativamente menor nessas organizações, esta pode ser uma explicação para a desigualdade de gênero nos salários \cite{barth2017effects}. Nos Estados Unidos, por exemplo, as mulheres compõem menos de 25\% dos trabalhadores nas ocupações STEM, ainda que representem aproximadamente 50\% do total de trabalhadores do país.