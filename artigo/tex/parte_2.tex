\section{Referêncial Teórico}
\subsection{Relação Anual de Informações Sociais (RAIS)}

A gestão pública brasileira, no setor trabalhista, possui um importante instrumento de coleta de dados denominado RAIS \cite{Sobre_a_RAIS}. Ele proporciona dados oficiais sobre o mercado de trabalho brasileiro. Tem por objetivo o suprimento às necessidades de controle da atividade trabalhista no País, o provimento de dados para a elaboração de estatísticas do trabalho e a disponibilização de informações do mercado de trabalho às entidades governamentais.

\subsection{Classificação Nacional de Atividades Econômicas (CNAE) e Classificação Brasileira de Ocupações (CBO)}

A CNAE \cite{Sobre_a_CNAE} é a classificação oficialmente adotada pelo Sistema Estatístico Nacional e pelos órgãos federais gestores de registros administrativos. Tem como aplicações: sistema estatístico, cadastro central de empresas, pesquisas econômicas estruturais, entre outros. Já a CBO \cite{Sobre_a_CBO} é um documento que retrata a realidade das profissões do mercado de trabalho brasileiro. 

Conforme os grupos CNAE que estão relacionados a atividades de TI, foram encontrados os códigos: Atividades dos serviços de tecnologia da informação (620) e Tratamento de dados, hospedagem na Internet e outras atividades relacionadas (631). Com esses grupos em mãos, foi possível encontrar suas respectivas ocupações, conforme podemos ver abaixo na tabela \ref{ocupacoes}.

\begin{table}[htbp]
	\caption{Ocupações analisadas}
	\begin{center}
		\begin{tabular}{|c|c|}
			\hline
			\textbf{Código} & \textbf{Descrição}                                  \\ 
			\hline
			212205           & Engenheiro de Aplicativos em Computacao               \\
			212210           & Engenheiro de Equipamentos em Computacao              \\
			212215           & Engenheiros de Sistemas Operacionais em Computacao    \\
			\hline 										
			212305           & Administrador de Banco de Dados                       \\
			212310           & Administrador de Redes                                \\
			212315           & Administrador de Sistemas Operacionais                \\
			212320           & Administrador em Segurança da Informação           \\
			\hline 									
			212405           & Analista de Desenvolvimento de Sistemas               \\
			212410           & Analista de Redes e de Comunicacao de Dados           \\
			212415           & Analista de Sistemas de Automacao                     \\
			212420           & Analista de Suporte Computacional                     \\
			\hline 									
			317105           & Programador de Internet                               \\
			317110           & Programador de Sistemas de Informacao                 \\
			317115           & Programador de Maquinas - Ferramenta com Comando Num. \\
			317120           & Programador de Multimidia                             \\
			\hline 									
			317205           & Operador de Computador (Inclusive Microcomputador)    \\
			317210           & Tecnico de Apoio ao Usuario de Informatica (Helpdesk) \\
			\hline
		\end{tabular}
		\label{ocupacoes}
	\end{center}
\end{table}

\subsection{Salário}

O salário é a remuneração que o empregado recebe pelo trabalho que realiza em uma instituição, pública ou privada. Segundo \cite{kryscynski2021firm}, os trabalhadores e suas empresas se envolvem em recorrentes relações de troca; de um lado, o trabalhador entrega valor, de outro, uma das recompensas que a empresa oferece em troca é o salário.
Considerando a relevância dessa temática na sociedade, é importante analisar a 
 desigualdade salarial, sobreduto entre grupos com alta dispersão salarial, seja em de nível educacional, raça, gênero etc. Este trabalho apresentará um diagnóstico com ênfase no sexo dentro dos vários cargos da área de TI, conforme apresentado na tabela \ref{ocupacoes}. 