\section{Referêncial Teórico}

\subsection{Relação Anual de Informações Sociais (RAIS)}

A gestão governamental do setor do trabalho conta com o importante instrumento de coleta de dados denominado de Relação Anual de Informações Sociais (RAIS) \cite{Sobre_a_RAIS}. Instituída pelo Decreto nº 76.900, de 23 de dezembro de 1975 e regida atualmente pelo Decreto nº 10.854, de 10 de novembro de 2021, a RAIS tem por objetivo: 

\begin{itemize}
	\item O suprimento às necessidades de controle da atividade trabalhista no País;
	\item O provimento de dados para a elaboração de estatísticas do trabalho;
	\item A disponibilização de informações do mercado de trabalho às entidades governamentais.
\end{itemize}

\subsection{A pandemia da COVID-19}

A pandemia da COVID-19, declarada no início de 2020 em virtude do coronavírus (Sars-Cov-2), deu início a uma nova crise de proporções globais. Em 31 de dezembro de 2019, a Organização Mundial de Saúde (OMS) foi alertada sobre o surgimento do vírus em Wuhan, na República Popular da China; entendido primeiramente como um caso de pneumonia, posteriormente descobriu-se a existência de um vírus específico. Em 11 de março de 2020, caracterizou-se a pandemia da COVID-19, que provocou milhões de mortes ao redor do mundo. 

Espalhando-se em escala global, a doença foi enfrentada de diversas maneiras, a depender da posição política das lideranças governamentais e do alinhamento com medidas sanitárias recomendadas pela OMS. Para evitar o aumento da contaminação, medidas de prevenção foram adotadas, como o uso de equipamentos de proteção individual (EPI), distanciamento e isolamento social. No Brasil, esse processo incluiu uma disputa judicial entre o governo federal e os governos estaduais. O primeiro foi contrário às medidas de isolamento social impostas por estados e municípios, com o argumento da necessidade de proteção da economia e dos empregos. O Supremo Tribunal Federal (STF) deu autonomia relativa a estados e municípios e permitiu a ação do governo federal especificamente para reforçar ações protetivas. Em se tratatando do gênero feminino no Brasil, percebeu-se que a pandemia teve impactos em muitas dimensões da vida das mulheres brasileiras. Dados apontaram aumento de 22\% nos casos de feminicídio no Brasil entre os meses de março e abril de 2020 e cerca de 7 milhões de mulheres deixaram seus postos de trabalho no início da pandemia, 2 milhões a mais do que o número de homens na mesma situação \cite{pandemia_impacta_mulheres}. A pandemia também afetou a saúde reprodutiva das mulheres e a sua saúde mental com a sobrecarga doméstica, os cuidados com os filhos e parentes \cite{pandemia_impacta_saude_mulheres}.

