\section{Referêncial Teórico}

\subsection{A pandemia da COVID-19}

A pandemia da COVID-19, declarada no início de 2020 em virtude do coronavírus (Sars-Cov-2), deu início a uma nova crise de proporções globais. Em 31 de dezembro de 2019, a Organização Mundial de Saúde (OMS) foi alertada sobre o surgimento do vírus em Wuhan, na República Popular da China; entendido primeiramente como um caso de pneumonia, posteriormente descobriu-se a existência de um vírus específico. Em 11 de março de 2020, caracterizou-se a pandemia da COVID-19, que provocou milhões de mortes ao redor do mundo. 

Espalhando-se em escala global, a doença foi enfrentada de diversas maneiras, a depender da posição política das lideranças governamentais e do alinhamento com medidas sanitárias recomendadas pela OMS. Para evitar o aumento da contaminação, medidas de prevenção foram adotadas, como o uso de equipamentos de proteção individual (EPI), distanciamento e isolamento social. No Brasil, esse processo incluiu uma disputa judicial entre o governo federal e os governos estaduais. O primeiro foi contrário às medidas de isolamento social impostas por estados e municípios, com o argumento da necessidade de proteção da economia e dos empregos. O Supremo Tribunal Federal (STF) deu autonomia relativa a estados e municípios e permitiu a ação do governo federal especificamente para reforçar ações protetivas. 

\subsection{A pandemia e as mulheres}

Em se tratatando do gênero feminino no Brasil, percebeu-se que a pandemia teve impactos em muitas dimensões da vida das mulheres brasileiras. Dados apontaram aumento de 22\% nos casos de feminicídio no Brasil entre os meses de março e abril de 2020 e cerca de 7 milhões de mulheres deixaram seus postos de trabalho no início da pandemia, 2 milhões a mais do que o número de homens na mesma situação \cite{pandemia_impacta_mulheres}. A pandemia também afetou a saúde reprodutiva das mulheres e a sua saúde mental com a sobrecarga doméstica, os cuidados com os filhos e parentes \cite{pandemia_impacta_saude_mulheres}.

\subsection{Relação Anual de Informações Sociais (RAIS)}

A gestão pública brasileira, no setor trabalhista, possui um importante instrumento de coleta de dados denominado RAIS \cite{Sobre_a_RAIS}. Ele proporciona dados oficiais sobre o mercado de trabalho brasileiro. Tem por objetivo o suprimento às necessidades de controle da atividade trabalhista no País, o provimento de dados para a elaboração de estatísticas do trabalho e a disponibilização de informações do mercado de trabalho às entidades governamentais.

\subsection{Classificação Nacional de Atividades Econômicas (CNAE) e Classificação Brasileira de Ocupações (CBO)}

A CNAE \cite{Sobre_a_CNAE} é a classificação oficialmente adotada pelo Sistema Estatístico Nacional e pelos órgãos federais gestores de registros administrativos. Tem como aplicações: sistema estatístico, cadastro central de empresas, pesquisas econômicas estruturais, entre outros. Já a CBO \cite{Sobre_a_CBO} é um documento que retrata a realidade das profissões do mercado de trabalho brasileiro. 

Conforme os grupos CNAE que estão relacionados a atividades de TI, foram encontrados os códigos: Atividades dos serviços de tecnologia da informação (620) e Tratamento de dados, hospedagem na Internet e outras atividades relacionadas (631). Com esses grupos em mãos, foi possível encontrar suas respectivas ocupações, conforme podemos ver abaixo na tabela \ref{ocupacoes}.

\begin{table}[htbp]
	\caption{Ocupações analisadas}
	\begin{center}
		\begin{tabular}{|c|c|}
			\hline
			\textbf{Código} & \textbf{Descrição}                                  \\ 
			\hline
			212205           & Engenheiro de Aplicativos em Computacao               \\
			212210           & Engenheiro de Equipamentos em Computacao              \\
			212215           & Engenheiros de Sistemas Operacionais em Computacao    \\
			\hline 										
			212305           & Administrador de Banco de Dados                       \\
			212310           & Administrador de Redes                                \\
			212315           & Administrador de Sistemas Operacionais                \\
			212320           & Administrador em Segurança da Informação           \\
			\hline 									
			212405           & Analista de Desenvolvimento de Sistemas               \\
			212410           & Analista de Redes e de Comunicacao de Dados           \\
			212415           & Analista de Sistemas de Automacao                     \\
			212420           & Analista de Suporte Computacional                     \\
			\hline 									
			317105           & Programador de Internet                               \\
			317110           & Programador de Sistemas de Informacao                 \\
			317115           & Programador de Maquinas - Ferramenta com Comando Num. \\
			317120           & Programador de Multimidia                             \\
			\hline 									
			317205           & Operador de Computador (Inclusive Microcomputador)    \\
			317210           & Tecnico de Apoio ao Usuario de Informatica (Helpdesk) \\
			\hline
		\end{tabular}
		\label{ocupacoes}
	\end{center}
\end{table}