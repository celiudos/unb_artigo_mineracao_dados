\section{Introduction}

With the exponential advancement of modern technologies, which are rapidly changing industries and societies globally, it is expected that the way people work, live, and interact with each other will be transformed at an unprecedented historical speed and scale \cite{hand1981artificial}. Technological innovations are indeed rapidly changing the boundary between tasks performed by people and machines, transforming the world of work \cite{aksoy2021robots}. What is not known is whether these transformations are also reshaping socioeconomic aspects such as gender wage gaps.

The protection of women’s work is ensured by the Consolidation of Labor Laws (CLT), of 1943. In addition, there is the Law 9.029, of 1999, which established rules on women’s access to the labor market. This device prohibits, for example, the announcement of job vacancies with reference to sex or that the person’s sex is a determining variable for remuneration and opportunities for professional advancement. However, the wage gap between men and women, which was trending downward until 2020, rose again in the country and reached 22\% at the end of 2022, according to data from the Brazilian Institute of Geography and Statistics (IBGE). This means that a Brazilian woman earns on average 78\% of what a man earns.

From the point of view of \cite{ahmed2015human}, discrimination in the labor market against women is evident, as when they have the same characteristics as men in relation to human capital and perform the same activity, they receive different remuneration due to their sex. In addition, the 2015 report from the United Nations (UN) shows that sex wage asymmetry in favor of men occurs in all industries and occupations in most of the 177 countries observed \cite{report2015onu}.

Within this context, this study proposes to verify whether this wage inequality is also perceived among professionals in the Information Technology (IT) field throughout the Brazilian territory, as well as to identify whether there is a difference in the number of layoffs. To do so, data from the Annual List of Social Information (RAIS) from 2015 to 2021 was used, which provides official data on the labor market in Brazil. 

Without detracting from the most varied gender identities, in this article sex data will be used indiscriminately, as the empirical data of the research is binary.

The work is organized as follows: Section II, which deals with the theoretical framework; Section III, presents related works; in Section IV, the methodology used and the steps to display the statistical data; in Section V, the results obtained are presented and, finally, in Section VI conclusion is presented.