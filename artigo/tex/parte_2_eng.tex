\section{Theoretical Framework}
\subsection{Annual list of Social Information (RAIS)}

Brazilian public management, in the labor sector, has an important data collection tool called RAIS\footnote{http://www.rais.gov.br/sitio/sobre.jsf}. It provides official data on the Brazilian labor market and is maintained by the Ministry of Economy. The objective of RAIS is to meet the needs of labor activity control in the country, provide data for the elaboration of labor statistics and make labor market information available to government entities. The great advantage of using this tool is that, in addition to providing official data, companies are legally obliged to send their data annually, including those of their employees.

\subsection{National Economic Activities Classification (CNAE) e Brazilian Classification of Occupations (CBO)}

CNAE\footnote{https://concla.ibge.gov.br/} is the classification officially adopted by the National Statistical System and federal agencies that manage administrative records. Its applications include: statistical system, central business registry, structural economic surveys, among others. On the other hand, CBO\footnote{https://empregabrasil.mte.gov.br/76/cbo/} is a document that portrays the reality of professions in the Brazilian labor market.

The codes of the CNAE groups related to IT activities are: Information technology service activities (620) and Data processing, web hosting and related activities (631). From these groups, it was possible to find the respective occupations, as shown in Table \ref{ocupacoes} below.

\begin{table}[htbp]
	\caption{Analyzed Occupations}
	\begin{center}
		\begin{tabular}{|c|c|}
			\hline
			\textbf{Code} & \textbf{Description}                                  \\ 
			\hline
			212205           & Engenheiro de Aplicativos em Computacao               \\
			212210           & Engenheiro de Equipamentos em Computacao              \\
			212215           & Engenheiros de Sistemas Operacionais em Computacao    \\
			\hline 										
			212305           & Administrador de Banco de Dados                       \\
			212310           & Administrador de Redes                                \\
			212315           & Administrador de Sistemas Operacionais                \\
			212320           & Administrador em Segurança da Informação           \\
			\hline 									
			212405           & Analista de Desenvolvimento de Sistemas               \\
			212410           & Analista de Redes e de Comunicacao de Dados           \\
			212415           & Analista de Sistemas de Automacao                     \\
			212420           & Analista de Suporte Computacional                     \\
			\hline 									
			317105           & Programador de Internet                               \\
			317110           & Programador de Sistemas de Informacao                 \\
			317115           & Programador de Maquinas - Ferramenta com Comando Num. \\
			317120           & Programador de Multimidia                             \\
			\hline 									
			317205           & Operador de Computador (Inclusive Microcomputador)    \\
			317210           & Tecnico de Apoio ao Usuario de Informatica (Helpdesk) \\
			\hline
		\end{tabular}
		\label{ocupacoes}
	\end{center}
\end{table}

\subsection{Salary}

Salary is the remuneration that an employee receives for the work they perform in an institution, whether public or private. According to \cite{kryscynski2021firm}, workers and their companies engage in recurring exchange relationships; on the one hand, the worker delivers value, and on the other, one of the rewards that the company offers in exchange is the salary.
Considering the relevance of this theme in society, it is important to analyze salary inequality, especially among groups with high salary dispersion, whether educational level, race, gender etc. This work will present a diagnosis with emphasis on gender within various positions in the IT area, as presented in Table \ref{ocupacoes}.

 \subsection{Information Technology}

 Information Technology (IT) is a set of activities that involves the use of technologies for obtaining, storing, accessing, processing, analyzing and disseminating information. IT is an area that is constantly growing and evolving; it has become increasingly essential for organizations, being one of the main factors of competitiveness, since it can be used to improve the efficiency and effectiveness of organizational processes, as well as enabling the creation of new products and services. With this importance, it is easy to see that technologies advance exponentially at an unprecedented speed and scale in history. Being synonymous with modernity, IT is an area that attracts many people, especially young people who are looking for a promising career. However, the technology area is fundamentally marked by male workers both worldwide and in Brazil \cite{de2021evidencias}, \cite{nunes2016genero}.

 According to \cite{de2021evidencias}, one of the explanations for the gender wage gap is the low female participation in occupations related to science, engineering and computing. People who hold positions in the STEM sector (Science, Technology, Engineering and Math) tend to have better salaries than average. The presence of engineers and scientists increases the productivity of organizations and increases their remuneration. If women have a relatively smaller participation in these organizations, this may be an explanation for gender inequality in wages \cite{barth2017effects}. In the United States, for example, women make up less than 25\% of workers in STEM occupations, even though they represent approximately 50\% of the country's total workforce.