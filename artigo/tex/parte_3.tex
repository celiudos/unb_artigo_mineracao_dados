\section{Trabalhos Relacionados}

No artigo \cite{de2021evidencias}, os autores analisam o hiato de gênero na indústria de transformação brasileira para diferentes quantis da distribuição de salários de homens e mulheres. Em todos os estratos de produtividade, verificou-se um diferencial de rendimentos desfavorável às mulheres.
Os resultados também sugerem que as mulheres com alto nível de
qualificação seriam atraídas para os setores mais produtivos, entretanto, como foi observado, esses mesmos setores apresentam uma baixa representatividade feminina nos cargos de chefia, aqueles que comumente oferecem maiores rendimentos.

Um dos estudos mais recentes realizados no tema em tela foi o de Ribeiro (2022) \cite{ribeiro2022efeitos}. Em sua tese de Douturado, a autora constatou que ocorre um aumento salarial de ambos gêneros,
sendo de 4,6\% para homens e de 5,2\% para mulheres, quando empregados em empresas com
alta intensidade de uso de TIs em relação a empresas com baixa intensidade de uso. O estudo focou em os dados da atividade econômica da empresa e de suas tecnologias utilizadas. Este trabalho apontou que os efeitos das TIs na desigualdade salarial são heterogêneos, com impactos distintos por nível educacional, mas não por sexo. 

Ainda no Brasil, um estudo realizado na região sul do país \cite{camargo2019relaccoes} analisou as relações de poder entre gêneros e a representatividade feminina das empresas de tecnologia da informação (TI) de Porto Alegre - RS. Verificou-se que, apesar de certos avanços de equidade entre gêneros, a desigualdade ainda estava presente nas organizações. Os resultados demonstraram que ainda existem diferenças de gênero significativas na área de TI, seja no acadêmico, que leva à desistência da maioria das mulheres, quanto no organizacional. 

O estudo de Hoisl e Mariani (2017) \cite{hoisl2017sa} analisou as disparidades salariais como base para determinantes dos desequilíbrios de gênero no mercado de trabalho. O estudo verificou que, mesmo com a capacitação e a educação como relevantes, as mulheres ganham menos que os homens. Ainda que contribuam para o desenvolvimento de invenções de alta qualidade tanto quanto os homens.

O presente estudo diferencia-se dos demais por analisar a diferença de salário e de desligamento entre homens e mulheres que trabalham especificamente na área de TI, não importando o ramo da organização ou se ela é do setor público ou privado, no período compreendido entre 2015 e 2021. O estudo apresenta um diagnóstico por gênero no território brasileiro, olhando para variáveis como salário, desligamento e escolaridade.
