\section{Trabalhos Relacionados}

Em 2022 foi realizado um estudo \cite{oliveira2022profissionais} que consistia em analisar novo segmento da Tecnologia da Informação (TI) no Nordeste brasileiro a partir das transformações e tendências trazidas pelos impactos da crise econômica desencadeada em 2015 e seu agravamento com a Pandemia da Covid-19. Este trabalho se detém, especialmente, sobre indicadores da condição laboral dos profissionais de TI na região. Para isso, utiliza-se de estudos empíricos – anteriormente realizados por nós e por terceiros – e de dados secundários. Quanto a estes últimos, priorizamos a Pesquisa Anual de Serviços (PAS) e a Pesquisa Nacional por Amostra de Domicílios Contínua (PNADC), ambas do Instituto Brasileiro de Geografia e Estatística (IBGE), e, com maior destaque, a Relação Anual de Informações Sociais (Rais). Os dados evidenciaram que, tanto no plano nacional como regional, a crise exerceu um efeito negativo sobre esse segmento, apesar do seu estoque de vínculos formais ter continuado sua trajetória de crescimento.

Outro estudo de 2022 \cite{cavalcanti2022reforma}, propôs uma analise dos impactos da Reforma Trabalhista de 2017 sobre os/as docentes do ensino básico privado de Pernambuco, especialmente sobre empregos, formas de contratação e remuneração. Para tanto, realizou-se um levantamento bibliográfico sobre Reforma, neoliberalismo e pandemia e utilizaram- -se técnicas quantitativas de manipulação de dados da Relação Anual de Informações Sociais – RAIS, do Cadastro Geral de Empregados e Desempregados – Caged e do Novo Cadastro Geral de Empregados e Desempregados – Novo Caged. A análise cobriu os períodos de 2010 a 2019, para os dados da RAIS, 2014 a 2019, para os dados do Caged, e 2020 a 2021, para os dados do Novo Caged. Destacam-se a alta rotatividade no mercado de trabalho da categoria, além do avanço de prerrogativas da Reforma sobre os contratos e desligamento.
  