\section{Trabalhos Relacionados}

Em 2001 foi realizado um estudo \cite{araujo2001diferenciais}, no Brasil, que consistia em analisar as diferenças de salários por gênero no Brasil. É importante destacar o período do estudo, pois já demonstra uma certa linha de pesquisa sobre este tema. No estudo, concluiu-se que o diferencial de salários entre homens e mulheres é mais relacionados ao salário intraocupacional, ou seja, dentro de uma mesma ocupação. Evidenciou-se que diferenças em atributos produtivos também explicaram uma parcela muito pequena do diferencial salarial, sugerindo tratamento desigual entre homens e mulheres. 

Um dos estudos mais recentes realizados foi o de Ribeiro (2022) \cite{ribeiro2022efeitos}. Em sua tese de Douturado, a autora constatou que ocorre um aumento salarial de ambos gêneros,
sendo de 4,6\% para homens e de 5,2\% para mulheres, quando empregados em empresas com
alta intensidade de uso de TIs em relação a empresas com baixa intensidade. O estudo focou em os dados da atividade econômica da empresa e de suas tecnologias utilizadas. Este trabalho apontou que os efeitos das TIs na desigualdade salarial são heterogêneos, com impactos distintos por nível educacional, mas não por sexo. 

Ainda no Brasil, um estudo realizado na região sul do país \cite{camargo2019relaccoes} analisou as relações de poder entre gêneros e a representatividade feminina das empresas de tecnologia da informação (TI) de Porto Alegre - RS. Verificou que, apesar de certos avanços de equidade entre gêneros, a desigualdade ainda estava presente nas organizações. Foram entrevistadas 10 mulheres que trabalham na área de tecnologia. Os resultados demonstraram que ainda existem diferenças de gênero significativas na área de TI, seja no acadêmico, que leva à desistência da maioria das mulheres, quanto no organizacional. 

O estudo de Hoisl e Mariani (2017) \cite{hoisl2017sa} analisou as disparidades salariais como base para determinantes dos desequilíbrios de gênero no mercado de trabalho. O estudo verificou que, mesmo com a capacitação e a educação como relevantes, as mulheres ganham menos que os homens. Ainda que contribuam para o desenvolvimento de invenções de alta qualidade tanto quanto os homens.

O presente artigo toma relevância devido à pesquisa na área de TI e o período analisado. Soma-se a importância da desigualdade de gênero em uma área predominantemente masculina, como a de TI, e a necessidade de se entender os fatores que levam a essa desigualdade.

% Em 2021, foi realizado um estudo por Rodolfo Bechtlufft e Bruno Lazarotti \cite{bechtlufft2021determinantes} que analisou os determinantes do diferencial de remuneração entre as carreiras do Poder Executivo do estado de Minas Gerais. Os autores utilizaram um modelo de regressão linear cujas variáveis explicativas foram elaboradas com base em diferentes abordagens teóricas acerca da determinação dos salários: teoria do capital humano, teoria da segmentação dos mercados, teoria da discriminação e sociologia das profissões, além de considerar as especificidades do setor público 1. Os resultados sugerem que a estrutura relativa de salários no setor público constitui uma manifestação concreta das diferenças de recursos políticos disponíveis às carreiras, do prestígio social das profissões e de aspectos estruturais da desigualdade de gênero, mas, diferentemente do que é proposto neste trabalho, o estudo focou apenas nas carreiras do poder executivo do estado de MG.
