\section{Concluding Remarks}

The objective of this work was to study the existence of salary differences and differences in the number of dismissals by gender in the IT sector in Brazil.

Observing the female population and the male population from the data that RAIS provides us, it was possible to perceive that the salary inequality between genders evidenced both in the 2022 IBGE survey and in the 2015 UN survey, already mentioned here, is also perceived among professionals in the Information Technology (IT) area throughout Brazilian territory to the detriment of women and independent of educational level, IT position and sector where they work, whether public or private. And this is a reality in all years of study, from 2015 to 2021.

The data also allowed us to show that this is a predominantly male profession, with a difference greater than 70\% in all years analyzed. Regarding the number of dismissals, it was not possible to perceive a significant difference between genders, except in 2021, when the percentage of dismissals of men was 35.82\% and women’s was 29.91\%.

The studied data does not allow us to clarify the causes of these differences, only helps us to evidence them. It is therefore necessary to deepen studies in order to identify the foundations of these disparities so that possible socio-economic and socio-educational measures can be pointed out in order to change this reality.

From the results obtained, it can be seen that there is scope for specific policies for increasing female participation in IT careers, which are considered one of the most promising careers today. Further studies should also suggest how much of the current segregation in the technology sector is generated by unequal treatment and how much is generated by individual preferences.




