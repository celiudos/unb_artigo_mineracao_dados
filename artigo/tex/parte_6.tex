\section{Conclusões}

O objetivo deste trabalho foi o de estudar a existência de diferenças salariais e diferenças na quantidade de demissões por gênero no setor de TI no Brasil.

Observando a população feminina e a população masculina a partir dos dados que a RAIS nos fornece, foi possível perceber que a desigualdade salarial e a taxa de desemprego entre gêneros evidenciadas tanto na pesquisa de 2022 do IBGE quanto na pesquisa de 2021 da ONU, aqui já citadas,  são também percebidas entre os profissionais da área de Tecnologia da Informação (TI) em todo o território brasileiro, ambas diferenças em desfavor da mulher e independentes do nível escolar e do setor onde trabalham, se público ou privado. E essa é uma realidade em todos os anos de estudo, de 2015 a 2021.

Os dados também nos permitiu evidenciar que esta é uma profissão predominantemente masculina, com uma diferença superior a 70\% a mais de homens em 2021.

Os dados estudados não nos permitem esclarecer a causa dessas diferenças, apenas nos ajudam a evidenciá-las. É, portanto, necessário aprofundar os estudos no sentido de identificar possíveis causas dessas disparidades para que possam ser apontadas possíveis providências socioeconômicas e socioeducativas no sentido de mudar essa realidade. 

Dos resultados obtidos, percebe-se que há escopo para políticas específicas de crescimento da participação feminina na carreira de TI, que é considerada uma das carreiras mais promissoras da atualidade. Maiores estudos devem sugerir também o quanto da atual segregação no setor de tecnologia é gerada pelo tratamento desigual e quanto é gerada pelas preferências individuais.




