\section{Conclusões}

A partir dos dados da Rais, foi possível realizar uma análise especificamente voltada ao foco deste estudo: os profissionais de TI. Em um primeiro momento, fixando-se nos dados do ano de 2019, ficou realçado o quanto o perfil dos profissionais de TI (referente a sexo, faixa etária, escolaridade e renda) destoa, para o Brasil, do perfil médio dos vínculos de empregos formais em geral, configurando-se aquele em um segmento claramente mais masculinizado, mais jovem, mais escolarizado e com acesso a rendas mais elevadas. Por outro lado, também sobressaíram os contrastes entre os profissionais de TI do Nordeste em relação ao Brasil, especialmente no que se refere à escolaridade e, ainda mais, à renda – em desfavor do primeiro, conta-se com nível escolar baixo e rendas reduzidas

Por último, ainda com base nos dados da Rais, agora nos utilizando de uma série histórica que abrange o período de 2010 a 2020, foi possível observar que, em contraste com o comportamento do estoque de vínculos formais de emprego em geral, para os âmbitos nacional e regional, o comportamento do estoque dos vínculos formais de emprego dos profissionais de TI não sofreu os impactos da crise e de seu prolongamento nos anos seguintes. Ao contrário, vinham e seguiram em uma trajetória de crescimento.

O mesmo, contudo, não pode ser dito no caso da remuneração. Sobre isso (tomando-se como medida o salário-mínimo), os dados evidenciaram que, tanto no plano nacional como regional, a crise exerceu efeito negativo. No caso do cenário nacional, os níveis de remuneração, que se encontravam estabilizados, caíram a partir de 2016. Quanto ao Nordeste, os níveis de remuneração vinham sofrendo um processo de desgaste que se acentuou a partir de 2016.